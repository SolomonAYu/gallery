% This is samplepaper.tex, a sample chapter demonstrating the
% LLNCS macro package for Springer Computer Science proceedings;
% Version 2.21 of 2022/01/12
%
\documentclass[runningheads]{llncs}
%
\usepackage[T1]{fontenc}
% T1 fonts will be used to generate the final print and online PDFs,
% so please use T1 fonts in your manuscript whenever possible.
% Other font encondings may result in incorrect characters.
%
\usepackage{graphicx}
% Used for displaying a sample figure. If possible, figure files should
% be included in EPS format.
%
% If you use the hyperref package, please uncomment the following two lines
% to display URLs in blue roman font according to Springer's eBook style:
%\usepackage{color}
%\renewcommand\UrlFont{\color{blue}\rmfamily}
%\urlstyle{rm}
%
\usepackage{multirow}
\usepackage{tabularx}
\usepackage{array}
\usepackage{float}
\usepackage{caption}
\usepackage{subcaption}

\makeatletter
\newcommand{\@chapapp}{\relax}%
\makeatother

%
% Packages for this paper
%
\usepackage[title]{appendix}
\usepackage[hyperfirst=false,shortcuts]{glossaries}
\usepackage[colorinlistoftodos,color=yellow!20,]{todonotes}
\usepackage{siunitx}
\sisetup{
  binary-units,
  per-mode=symbol,
  group-separator={,},
  group-minimum-digits=4
}

%
% User-defined commands
%
\newcounter{tbdcounter}
\setcounter{tbdcounter}{0}
\newcommand{\TBD}[1]{\addtocounter{tbdcounter}{1}{[\bf TBD\arabic{tbdcounter}}:
{\it #1\/}{\bf ]}}

%
% User-defined SI units
%
\DeclareSIUnit{\tuple}{\text{T}}

% Configuration for glossaries
\setacronymstyle{long-short}

%
% Labels of glossaries and acronyms.
% In many cases, we can use gls- commands and ac- commands interchangeably.
%
\newglossaryentry{C++}{name={\makebox{C\hspace{-.04em}\raisebox{.2em}{\tiny\bf
+}\hspace{-.06em}\raisebox{.2em}{\tiny\bf +}}},description={C++}}
\newglossaryentry{YhSB}{name={Yahoo streaming benchmark},description={Yahoo
Streaming Benchmark}}
\newacronym[plural=SPEs,firstplural=stream processing engines]{SPE}{SPE}{stream
processing engine}
\newacronym{NUMA}{NUMA}{non-uniform memory access}
\newacronym{SIMD}{SIMD}{single instruction multiple data}
\newacronym{HLS}{HLS}{heterogeneous lookahead scheduling}
\newacronym{RDMA}{RDMA}{Remote Direct Memory Access}
\newacronym{RLAS}{RLAS}{Relative-Location Aware Scheduling}
\newacronym{JVM}{JVM}{Java Virtual Machine}

\glsdisablehyper
\begin{document}
%
\title{Leveraging both worlds: Scale-up techniques for scale-out data stream
processing}
%
\titlerunning{Internship Report, Winter 2024}
% If the paper title is too long for the running head, you can set
% an abbreviated paper title here
%
\author{Junhao~Zhu\orcidID{0009-0006-2954-4169}}
%
\authorrunning{J.~Zhu}
% First names are abbreviated in the running head.
% If there are more than two authors, 'et al.' is used.
%
\institute{
Department of Computer Science and Engineering, Yonsei University\\
50 Yonsei-ro, Seodaemun-gu, Seoul, Republic of Korea 03722\\
%\email{lncs@springer.com}\\
\email{2021147513@yonsei.ac.kr}
%\url{https://cs.yonsei.ac.kr/index.php}
}
%
\maketitle              % typeset the header of the contribution
%
\begin{abstract}
Big data stream processing emerged in academics and industry more than 20 years
ago, and it has seen an increasing demand for its capability to process and
analyze data in real time. Java-based streaming processing engines provide a
platform that focuses on programmer productivity and accelerates scale-out
deployment of real-time streaming analytics. However, stream processing engines
are confronted with performance challenges because the current throughput
performance does not fully utilize a single-node memory bandwidth, but the
throughput is substantially limited by inefficient scale-out actors that
include network communication channels between inter-node actors and
inter-process communication channels between actors on the same node. We
present a hybrid approach that balances the requirements and advantages of both
worlds: vertical scaling and horizontal scaling. First, we eliminate
inter-process communication channels and replace them with shared-memory
buffers. Second, we minimize the number of actors by fusion of streaming
operators. Third, we eliminate the global key-value store and replace it with
our non-blocking global store, which preserves the semantics of the target
workload. Lastly, we leverage the scale-out deployment of a stream topology on
a multi-node streaming processing engine. We show that our hybrid solution
significantly increases the throughput performance of a stream processing
engine. We compare our results with existing approaches, validating our
approach's scalability vertically and horizontally.
%The abstract should briefly summarize the contents of the paper in
%150--250 words.

\keywords{Stream Processing Engines \and Cloud Computing \and Performance
Measurements and Analysis}
\end{abstract}
%
%
%
\section{Introduction}
Scale-out data stream processing has been a leading technology for big data
real-time analytics, with abundant availability of commodity cloud nodes such
as Google Compute Engine and Amazon EC2. Early and recent
\acp{SPE}~\cite{SparkStreaming2012,Storm2014,Flink2015,KafkaStreams2021} share
\ac{JVM}-based architecture in common with prior batch processing engines,
adopting the cloud scalability and leveraging the existing cloud software
stacks such as Kafka and Redis.

%\section{Background}
%TBD.

\section{Scale-up Optimization Techniques}

\subsection{NUMA and Cache Locality}
Several researches have succeeded in improving the throughput of \ac{SPE} by
utilizing modern hardware. Optimization techniques that fall in this 
category also benefit multi-nodearchitecture. Memory access often become the 
bottleneck of stream processing. \emph{StreamBox}~\cite{StreamBox}, \emph{BriskStream}~\cite{BriskStream} and \emph{Grizzly}~\cite{Grizzly}
attempt to maximize the \ac{NUMA} locality of the target system by explicitly
pinning streams to a \ac{NUMA} region to reduce cross-\ac{NUMA} communication.
%
Grizzly presents an adaptive query compilation technique.  It detects the
underlying \ac{NUMA} configuration and optimizes window aggregation by
processing values inside a \ac{NUMA} region and merging those values between
\ac{NUMA} regions only at each window end.
%
Their technique improves the processing bandwidth significantly because it
minimizes the communication overhead between \ac{NUMA} regions.
%
BriskStream presents a \ac{NUMA}-aware streaming execution plan optimization,
\emph{\ac{RLAS}}. A producer and a consumer on different sockets or on the same
socket are examples of relative locations. 
%
\Ac{NUMA} distance influences the performance of an operator and makes it
dependent to the current execution plan. Scheduling an operator to a socket may
prohibit other operators to be scheduled to the same socket. Therefore,
BriskStream introduces placement optimization heuristics to reduce the solution
space.
%
%That is, NUMA-awareness facilitates optimized scheduling by introducing an
%additional variable in terms of capability. However, the solution space
%becomes excessively large due to dependencies on all previous decisions, which
%might conflict with each other.
Beside NUMA-awareness, increasing data locality reduces performance penalty.
%
For example, Grizzly uses hardware performance information such as the number
of cache misses to determine the data characteristics and migrate the code
based on the information.
\emph{Saber}~\cite{SABER} load tuples from the same window into a GPGPU work
group's cache memory to speed up serialization and deserialization.

\subsection{SIMD parallelism in GPGPUs}
Saber proposes to use the GPGPU computing resource for stream data processing.
It takes advantage of GPGPU accelerators and their \ac{SIMD} parallelism. Saber
is a hybrid stream processing engine that employs a \emph{\ac{HLS}} algorithm,
which assigns tasks to either GPGPU or CPU based on recent query task
throughput.
%
GPGPU follows the \ac{SIMD} model: a core executes the same operation on
multiple data in each cycle. This allows the control logic to be simple.  As a
result, while CPU suffers from the complexity of query operators, GPGPU is
bounded by the data movement overhead. Saber also concluded CPU is faster when
the selectivity, the fraction of rows chosen by the predicate, is low. On the
contrary, GPGPU is faster with higher selectivity.


\subsection{Shared Memory Communication}
\Acp{SPE} intensively use network connections to deliver tuples
from one node to another. \Acp{SPE} designed for a sigle node take advantage
of shared-memory communication.
%
An operator in a stream pipeline may contain a blocking logic that requires the
materialization of its result to some extent for the next operator. It needs to
store the result before each emission. Zeuch~et al.~\cite{Zeuch2019} state that
full materialization is impractical because streaming data is unbounded.
Therefore partially materialized data is used instead. Then, they suggest that
in constrast to interpretation-based model, operator fusion allows us to omit
queues for message passing.
They propose two strategies to distribute processing among computing units. We
explain them in Section~\ref{sec:merge}.
%
Using their memory optimizations, their single-node \ac{C++} implementation
achieves a maximum throughput of \SI{384}{\mega\tuple\per\second}, near memory
bound.

\subsection{Adaptive Query Compilation}
To achieve better resource utilization, Grizzly~\cite{Grizzly} introduces query
compilation which fuses operators at compile time, and performs all operators
of a pipeline over a chunk of input records during query execution. Compilation
into compact code fragments allows better utilization of registers and enables
memory access via raw pointer. While code is statically generated, Grizzly is
still able to exploit dynamic workload by adding a feed back loop between code
generation and query execution. In details, there are four phases during query
execution, each applies more aggressive optimization than the previous phase,
except for the last profiling and adaptive optimization phase which provides a
coarse-grained induction about data characteristics for Grizzly to re-optimize
and generate new code variant.

\subsection{Data Synchronization\label{sec:merge}}
When the tuples need to be buffered, a queue is necessary at either the inbound
of consumer or the outbound of producer. Zeuch~et al.~\cite{Zeuch2019} refer
the former as upfront partitioning which is used by Flink, Spark, and Storm. In
this case, data is shuffled and producers push the data to disjoint set of
responsible consumers. On the other hand, the latter can be further divided
into local merge and global merge distinguished by the timing of
synchronization. Both local merge and global merge require consumers to fetch
the data by their own. While local-merge producers store partial results
locally and merge into global data structure when necessary, global-merge
producers continuously synchronize during execution. Hazelcast
Jet~\cite{Hazelcast} explicitly mentions they are using local merge approach,
where local partial results are followed by global combining.

\subsection{Coroutine-based Execution} 
Hazelcast Jet~\cite{Hazelcast} does not spawn a thread for each operator.
Instead, it deploys the complete dataflow graph on every available core. This
prevents dataflow spanning over multiple machines therefore reducing network
overhead. In addition, running tasks in single threaded loop and making them 
voluntarily yield control avoids expensive operating system context switches.
It allows execution thread to remain on the same CPU core for longer time
periods, so that CPU cache lines are preserved. In this case, a task should
never block otherwise all tasks sharing the execution thread would be
influenced. Tasks communicate via shared-memory, more specifically wait-free
single-producer-single-consumer queues.

\section{Scale-out Optimization Techniques}

\subsection{Comparison of Experimental Setups}

\begin{table}[ht]
  \centering
  \begin{tabular}{|r|l|}
  \hline
  & Flink
  \\ \hline
  system version & Apache Flink 1.2.1
  \\ \hline
  hardware setups & 10 Kafka brokers with 2 partitions each
  \\ \hline
   & 10 compute machine
  \\ \hline
   & 1 Xeon E3-1230-V2@3.30GHz CPU 
   \\
   & and 32GB RAM
   \\ \hline
   & 10 GigE Ethernet between compute nodes 
   \\ \hline
   & 1GigE Ethernet between Kafka 
   \\
   & cluster and Flink/Storm nodes
   \\ \hline
   \# of cores per each node & 4
   \\ \hline
   \# of the ad campaigns & 100 distinct
   \\ \hline
   freq written into key-value table & 100 window per sec
   \\ \hline
   freq of final windows written & once per hour
   \\
   to the database &
   \\ \hline
   data generation & per-gen in kafka
   \\ \hline
   To query the window contents & Akka actor system
   \\ \hline
  \end{tabular}
  \caption{Ververica Experimental Setup}
  \end{table}
  
  \begin{table}[ht]
  \centering
  \begin{tabular}{|c|c|c|c|}
  \hline
  & Spark & KafkaStreams & Apache Flink
  \\ \hline
  system version & Runtime 3.1 & 0.10.2.1 & 1.2.1
  \\ \hline
  single-node version & 1 (c3.xlarge) & & 
  \\ \hline
  \# of worker nodes & 10 (r3.xlarge) & 10 (r3.xlarge) & 10 (r3.xlarge)
  \\ \hline
  \# of cores per node & 4 & 4 & 4
  \\ \hline
  & pre-generate & pre-generate & pre-generate
  \\
  data generation & in-memory & in Kafka & in-memory
  \\ \hline
  & in-memory & static table & in-memory
  \\
  key-value store (Redis) & (DataSet) & (KTable) & (HashMap)
  \\
  ad/campaign & does not affect & does not affect & 10 or 1
  \\ \hline
  & & & periodically log
  \\
  throughput & Streaming Query & & the number of records
  \\
  calculation & Listener & & processed to a log
  \\ \hline
  \end{tabular}
  \caption{Databricks Experimental Setup}
  \end{table}

\subsection{Experimental Results}
\subsubsection{Ververica}
%
In the context of the \gls{YhSB}, two variants were compared to showcase the
effectiveness of a new approach. The first variant involved \num{1000000} 
campaigns with windows stored and updated directly in the key-value store. 
However, this approach led to the key-value store becoming a bottleneck at 
around \SI{280000}{\tuple\per\second}, requiring significant scaling or 
elimination of the store. The second variant addressed large windows and 
numerous distinct keys, with a streaming application updating
millions of entries per second in the key-value store. The developers of 
Ververica proposed a new architecture to eliminated the key-value store bottleneck, 
achieving a throughput of \SI{15000000}{\tuple\per\second}. 
%
In experiments with Flink on Databricks Runtime-3.4, Vererica developers utilized Spark-2.2 
with some 2.3 functionality and Apache Flink-1.2.1, and achieved the following throughput results: 
Compared to Spark's \SI{2500000}{\tuple\per\second}, Flink achieves a higher throughput 
of \SI{4000000}{\tuple\per\second} compared to Spark's \SI{2500000}{\tuple\per\second}
and records consistent performance as the number of advertisements increases from \num{1} to \num{10}.
%
\subsubsection{Databricks}
%
In attempting to replicate Flink's performance results, Databricks researchers made the following observations: 
Flink achieved \SI{16000000}{\tuple\per\second} throughput on Databricks using commercial cloud hardware when generating
one ad per campaign, but throughput dropped to only \SI{1000000}{\tuple\per\second}
when generating \num{10} ads per campaign. In the final benchmark setup in Databricks Community Edition, Spark's throughput
was 1.5 times that of Flink at \SI{2500000}{\tuple\per\second}. The benchmark was scaled up to a total of \num{40} cores on 
\num{10} worker nodes per system, and the results showed that Spark's throughput was 2.9 times
that of Flink reaching \SI{60000000}{\tuple\per\second}.

\subsection{Actors}
Actor fusion is a concept in Akka where multiple actors are merged or combined into a single actor. 
This is usually done for performance optimization and better resource utilization. Ververica developers 
used Akka to create an actor system to support a queryable state, allowing external systems to query 
the state of the Flink application through Akka actors. In conclusion, Actors act as an entry point for 
handling state queries in a distributed system. Also, manages the interaction with the “RetrievalService” for 
actor URL retrieval and ensures fault tolerance through failover mechanisms, caching, and periodic cache refresh.

\subsection{Data Generator}
\subsubsection{Databricks}

To solve the global store bottleneck, the experiment chose to eliminate the dependence on external store systems
such as Redis. This leads to the need for internal data generators to produce the data required for system operation
while avoiding the communication overhead with external store systems. In-memory joins or hashmap lookups were 
used as alternatives in the experiments to generate test data. This approach might include dynamically building 
data structures in memory or leveraging hashmap for fast data retrieval to better suit the needs of 
stream processing systems.

\subsubsection{Ververica}
The Flink window operator acts as a Kafka consumer to ingest incoming streaming data. 
Flink's windowing mechanism allows the efficient handling of data within these specified 
windows and based on time intervals.

\subsection{Native Window Operators}
Researchers of Ververica have developed a custom operator for Flink, designed to compute windows
and enable querying the state externally from the streaming system through Akka. 
It includes the capability to start a registration service and initialize the Akka Actor system.
Once registered, it can handle external query requests. 
%
Additionally, the operator is equipped to process elements from the input stream, 
determining which window an element belongs to and updating the corresponding key's count and access time in the window.
It also asynchronously saves the operator's state and other window information to the state backend during checkpointing,
returning an asynchronous state handle. In summary, this custom operator facilitates window computation, 
and state querying through Akka, and integrates seamlessly with Flink's checkpointing mechanism, 
showcasing Ververica's research contributions to Flink's functionality.
\subsection{Eliminate key-value bottlenecks}
\subsubsection{Ververica}
%
For stream processing applications with large windows and a large number of different key-value pairs, 
the traditional approach is to periodically write the data in the window to a key-value store for querying. 
However, this can result in a large number of write and query operations, increasing the load on the database. 
The Ververica researchers' solution was to directly expose the real-time window as queryable, 
avoiding the overhead of periodic writes to the database, while using Akka actors to implement real-time queries. 
%
In traditional scenarios, the database can easily become a performance bottleneck when processing 
a large number of events. In order to solve this problem, the article mentioned using Flink's window operator 
and local state to replace the traditional database. Flink window operators allow state to be maintained locally, 
thus avoiding communication overhead with remote databases and improving processing speed.

\subsubsection{Databricks}
%
The benchmark is designed to ensure that the streaming system itself is the bottleneck, not an external service. 
To achieve this goal, the authors removed Redis from the original \gls{YhSB} to simplify the benchmark 
and focus on measuring the performance of the streaming system. For Flink and Spark, data is generated in memory. 
For Kafka Streams, data needs to be persisted in Kafka, the data is generated using Spark and then written to Kafka.

\subsection{Network Communication}
Inter-node communication is often conducted using
traditional TCP/IP transport layer protocol. Zeuch~et al suggest that \ac{RDMA}
enables direct access to memory of remote machine bypassing the remote CPU.
This can be beneficial in three ways.
\begin{enumerate}
\item Remote CPU resource is not consumed
\item Remote CPU cache is not polluted
\item Traffic does not go through software network stack
\end{enumerate}

%\section{Related Work}
%TBD.

%\section{Conclusions}
%TBD.


\begin{credits}
\subsubsection{\ackname}
This is acknowledgment.

\end{credits}
%
% ---- Bibliography ----
%
% BibTeX users should specify bibliography style 'splncs04'.
% References will then be sorted and formatted in the correct style.
%
\bibliographystyle{splncs04}
% \bibliography{mybibliography}
%
\bibliography{references.bib}

\iffalse
\clearpage
\newpage
\begin{subappendices}
\renewcommand{\thesection}{\arabic{section}}
\section{Analyzing efficient stream processing on modern hardware \cite{Zeuch2019}}
\subsection{Strategy 1: Upfront Partitioning}
Used by Flink, Spark, and Storm, upfront partitioning is a strategy that shuffles intermediate data among n producers and m consumers. Producers are mapped to disjoint set of responsible consumers.

\subsection{Strategy 2: Local Merge}
Consumers pull data by their own. Each worker stores the partial results locally, and those results are merged into global data structure when data is required by pipeline-breaker.

\subsection{Strategy 3: Global Merge}
In contrast to local merge, global merge does not need merging step at the end. All workers synchronize their result to the global data structure during execution.

\end{subappendices}
\fi

\end{document}
